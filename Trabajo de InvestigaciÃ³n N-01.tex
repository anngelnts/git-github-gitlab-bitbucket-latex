% Generated by GrindEQ Word-to-LaTeX 
\documentclass{article} %%% use \documentstyle for old LaTeX compilers

\usepackage[english]{babel} %%% 'french', 'german', 'spanish', 'danish', etc.
\usepackage{amssymb}
\usepackage{amsmath}
\usepackage{txfonts}
\usepackage{mathdots}
\usepackage[classicReIm]{kpfonts}
\usepackage[dvips]{graphicx} %%% use 'pdftex' instead of 'dvips' for PDF output

% You can include more LaTeX packages here 


\begin{document}

%\selectlanguage{english} %%% remove comment delimiter ('%') and select language if required


\noindent 

\noindent 

\noindent 

\noindent 

\noindent SISTEMAS DE CONTROL DE VERSIONES\includegraphics*[width=0.47in, height=0.63in, keepaspectratio=false]{image1}

\noindent 
\[6\] 


\noindent \textbf{\includegraphics*[width=1.08in, height=1.46in, keepaspectratio=false]{image2}}

\noindent \textbf{}

\noindent \textbf{UNIVERSIDAD PRIVADA DE TACNA}

\noindent \textbf{}

\noindent \textbf{FACULTAD DE INGENIERIA}

\noindent \textbf{\textit{}}

\noindent \textbf{Escuela Profesional de Ingenier\'{i}a de Sistemas}

\noindent \textbf{}

\noindent \textbf{}

\noindent 

\noindent \textbf{ TRABAJO DE INVESTIGACION Nº 01}

\noindent \textbf{\textit{}}

\noindent \textbf{}

\noindent Curso: Base de Datos 2

\noindent \textbf{\textit{}}

\noindent \textbf{}

\noindent \textbf{}

\noindent Docente: Ing. Patrick Cuadros

\noindent \textbf{\textit{}}

\noindent \textbf{}

\noindent Integrantes:

\noindent \textbf{\textit{}}

\noindent 

\noindent \textbf{Moreno Mulluni, Luis Angel     (2017057864)}

\noindent \textbf{\textit{ }}

\noindent 

\noindent 

\noindent \textbf{Tacna -- Per\'{u}}
\[2018\] 
\textbf{\textit{}}

\noindent \textbf{}

\noindent \textbf{\eject }

\noindent \textbf{GitHub}

\noindent GitHub~es una~plataforma de desarrollo colaborativo para alojar proyectos utilizando el sistema de~control de versiones~Git. Se utiliza principalmente para la creaci\'{o}n de c\'{o}digo fuente de programas de computadora. El software que opera GitHub fue escrito en~Ruby on Rails. Desde enero de~2010, GitHub opera bajo el nombre de~GitHub, Inc.~Anteriormente era conocida como~Logical Awesome LLC. El c\'{o}digo de los proyectos alojados en GitHub se almacena t\'{i}picamente de forma~p\'{u}blica, aunque utilizando una cuenta de pago, tambi\'{e}n permite hospedar repositorios privados.

\noindent El 4 de junio de 2018,~Microsoft~compr\'{o} GitHub por la cantidad de 7.500 millones de d\'{o}lares.

\noindent \textbf{Caracter\'{i}sticas}

\begin{enumerate}
\item \textbf{ }Wiki~para cada proyecto.

\item  P\'{a}gina web~para cada proyecto.

\item  Gr\'{a}fico para ver c\'{o}mo los desarrolladores trabajan en sus repositorios y~bifurcaciones~del proyecto.

\item  Funcionalidades como si se tratase de una~red social, por ejemplo, seguidores.

\item  Herramienta para~trabajo colaborativo~entre programadores.

\item  Gestor de proyectos de estilo~Kanban.
\end{enumerate}

\noindent \textbf{Beneficios}

\noindent Existen muchas caracter\'{i}sticas que justifican la popularidad de GitHub. A continuaci\'{o}n, alguno de ellos:

\begin{enumerate}
\item  \textbf{Seguimiento de errores.} Esta caracter\'{i}stica pertenece a las funciones de colaboraci\'{o}n y permite mejorar la calidad del c\'{o}digo al mantener registro de los errores de software detectados en el proyecto.

\item  \textbf{B\'{u}squeda r\'{a}pida:} El repositorio proporciona una estructuraci\'{o}n conveniente de proyectos que permite una b\'{u}squeda y clasificaci\'{o}n eficiente. Adem\'{a}s, una indexaci\'{o}n adecuada permite a los usuarios encontrar cualquier cadena de c\'{o}digo en los repositorios p\'{u}blicos.

\item  \textbf{Comunidad:} Esta enorme comunidad de desarrolladores en todo el mundo es una gran fuente de experiencia y habilidades compartidas.

\item  \textbf{Compartir:} El c\'{o}digo fuente del proyecto no s\'{o}lo se puede copiar con Git, tambi\'{e}n se puede descargar como archivo.

\item  \textbf{Trabajo conjunto:} GitHub brinda funciones eficientes para la administraci\'{o}n de equipos.

\item  \textbf{Compatibilidad:} Los proyectos con el c\'{o}digo en GitHub se pueden personalizar f\'{a}cilmente a cualquier servicio host en la nube.

\item  Admite importaci\'{o}n con Git, SVN, TFS.
\end{enumerate}

\noindent \textbf{Desventajas}

\begin{enumerate}
\item \textbf{ El servicio no es completamente gratuito}. Para acceder a todas las funciones de GitHub, se debe actualizar a un usuario Premium.

\item  \textbf{Limitaciones de tama\~{n}o}. Los archivos no pueden ser mayores a 100 MB mientras que el repositorio puede alojar 1 GB de informaci\'{o}n.
\end{enumerate}

\noindent 

\noindent \textbf{GitLab}

\noindent Gitlab~es un servicio web de control de versiones y desarrollo de software colaborativo basado en~Git. Adem\'{a}s de gestor de repositorios, el servicio ofrece tambi\'{e}n alojamiento de~wikis~y un~sistema de seguimiento de errores, todo ello publicado bajo una~Licencia de c\'{o}digo abierto.

\noindent Fue escrito por los programadores~ucranianos~Dmitriy Zaporozhets y Valery Sizov en el lenguaje de programaci\'{o}n~Ruby. La compa\~{n}\'{i}a, GitLab Inc., cuenta con un equipo de 150 miembros1 y m\'{a}s de 1400 usuarios. {} Es usado por organizaciones como la~NASA, el~CERN,~IBM~o~Sony.

\noindent \textbf{Beneficios}

\begin{enumerate}
\item \textbf{ }Es gratis. Eso significa que los usuarios pueden tener un n\'{u}mero ilimitado de repositorios privados. Esto es la versi\'{o}n comunitaria, y los usuarios tendr\'{a}n que pagar si requieren la versi\'{o}n empresarial. Esta \'{u}ltima agrega algunas caracter\'{i}sticas adicionales a la funcionalidad b\'{a}sica que mejora la interacci\'{o}n con herramientas en l\'{i}nea, flujo de trabajo y administraci\'{o}n de servidores, entre otras.

\item  GitLab opera bajo una licencia de~c\'{o}digo abierto.

\item  Seguimiento de errores y edici\'{o}n de c\'{o}digo basado en la web.

\item  Integraci\'{o}n con LDAP que permite localizar y acceder a diversos recursos de internet. GitLab EE soporta varios servicios LDAP y sincronizaci\'{o}n de grupos.

\item  Soporta la importaci\'{o}n de Git.
\end{enumerate}

\noindent \textbf{Desventajas}

\begin{enumerate}
\item \textbf{ }Interface relativamente lenta

\item  Frecuentes problemas t\'{e}cnicos con los repositorios
\end{enumerate}

\noindent 

\noindent \textbf{Bitbucket}

\noindent El servicio tambi\'{e}n es muy similar a GitHub y refleja la mayor\'{i}a de sus caracter\'{i}sticas con ligeras diferencias. Bitbucket est\'{a} mejor orientado a los equipos de desarrollo profesional, ya que proporciona grandes beneficios para ellos, como repositorios privados gratuitos, integraci\'{o}n con Jira, revisi\'{o}n de c\'{o}digo avanzado~ y CI/CD (Integraci\'{o}n Continua/Despliegue continuo) integrado. Al mismo tiempo, con el crecimiento del equipo, Bitbucket ofrece condiciones de precios m\'{a}s adecuadas comparadas con GitHub y GitLab. Bitbucket tambi\'{e}n proporciona una modelo de implementaci\'{o}n flexible para equipos.

\noindent \textbf{Beneficios}

\begin{enumerate}
\item \textbf{ }Repositorios privados para equipos peque\~{n}os. Equipos peque\~{n}os, hasta 5 integrantes, pueden obtener un n\'{u}mero ilimitado de repositorios y 500 minutos de compilaci\'{o}n.

\item  Bitbucket tiene una b\'{u}squeda consistente de c\'{o}digo. En lugar de simplemente indexar su c\'{o}digo como texto, Butbucket utiliza la b\'{u}squeda sem\'{a}ntica que analiza la sintaxis de su c\'{o}digo, asegurando que las definiciones que coincidan con su t\'{e}rmino de b\'{u}squeda tengan prioridad sobre los usos y nombres de las variables.

\item  El enfoque empresarial de Bitbucket significa que encontrar\'{a} funciones como la lista blanca de IPs y verificaci\'{o}n en dos pasos para dar a los administradores mayor control sobre qui\'{e}n puede ver, enviar o clonar un repositorio de c\'{o}digo privado.

\item  Bitbucket viene con Trello para que pueda comenzar con un gran rastreador de problemas gratuito o se puede aprovechar la existencia de una instancia de Jira. Al ser propiedad de Atlassian, Jira y Bitbucket, se integran en cada etapa del desarrollo, desde la creaci\'{o}n hasta la implementaci\'{o}n. Con componentes de seguimiento de errores integrados, Jira actualiza autom\'{a}ticamente la informaci\'{o}n sobre el problema detectado.

\item  Importaci\'{o}n de proyectos Git existentes desde Excel, Github, entre otros.

\item  Condiciones especiales para estudiantes y profesores.

\item  Compatible con importaciones de Git, CodePlex, Google Code, SVN.
\end{enumerate}

\noindent \textbf{Desventajas}

\begin{enumerate}
\item \textbf{ }No es de c\'{o}digo abierto, pero admite proyectos de c\'{o}digo abierto.
\end{enumerate}

\noindent 

\noindent \textbf{SourceForge}

\noindent Se trata del inicio de todo (en lo que a este tema respecta, claro),~el padre de todos los servicios de alojamiento para el control de versiones online. Dat\'{a}ndose en 1999 fue el primer sitio web open source de este tipo. El c\'{o}digo fuente del software de SourceForge estaba a disposici\'{o}n del p\'{u}blico durante los primeros a\~{n}os, hasta que en 2001 no se volvieron a lanzar revisiones oficiales del mismo.

\noindent No obstante, con el paso de los a\~{n}os y a pesar de no liberar nuevas versiones de su c\'{o}digo, sus desarrolladores no han dejado de a\~{n}adir funcionalidades m\'{a}s all\'{a} del sencillo gestor de versiones, como son el seguimiento de errores, Pull Request, mirrors para una gesti\'{o}n de descarga m\'{a}s veloz{\dots} Proporcionando al p\'{u}blico y sus proyectos un alojamiento web gratuito y muy eficaz.

\noindent \textbf{Resumen}

\begin{enumerate}
\item \textbf{ }El primer hosting para el control de versiones.

\item  S\'{o}lo aloja proyectos open source.

\item  Seguimiento de fallos.

\item  Pull Request.

\item  Soporte para CVS, SVN, Git y Mercurial.
\end{enumerate}

\noindent 

\noindent \textbf{¿Qu\'{e} caracter\'{i}sticas tiene un buen repositorio?}

\noindent Un sistema de control de versiones puede ser de tres tipos:

\begin{enumerate}
\item  \textbf{Local:} Todos los desarrolladores se encuentran en el mismo sistema de archivos.

\item  \textbf{Centralizado:} Se tiene una copia del proyecto en un servidor central y los integrantes del equipo realizan una actualizaci\'{o}n de estos archivos de acuerdo a los cambios que realizan.

\item  \textbf{Distribuido:} Los desarrolladores trabajan en su repositorio local y los cambios se actualizan entre repositorios.
\end{enumerate}

\noindent Antes de continuar, veamos qu\'{e} caracter\'{i}sticas hacen a un buen repositorio:

\begin{enumerate}
\item  Pull request: Cuando los usuarios realizan cambios en el c\'{o}digo fuente y lo env\'{i}an al repositorio, se les notifica a sus colaboradores para que analicen y revisen los cambios.

\item  Revisi\'{o}n de c\'{o}digo web

\item  Edici\'{o}n: Si un repositorio tiene la posibilidad de sugerir una edici\'{o}n colaborativa en tiempo real, agrega mucho a la calidad del repositorio.

\item  Seguimiento de errores: Todos los proyectos tienen errores. As\'{i} que es genial si un repositorio permite rastrear y resolver errores de forma colaborativa.

\item  Autenticaci\'{o}n two-factor para garantizar la protecci\'{o}n de las cuentas de usuario.

\item  La capacidad de crear bifurcaciones o clones del repositorio.

\item  La posibilidad de compartir segmentos de c\'{o}digo o archivos con cualquier persona que lo requiera.

\item  Integraci\'{o}n con servicios de terceros.

\item  Importaci\'{o}n de repositorios. Si los usuarios cambian de un servicio a otro, es una buena idea que los repositorios permitan importar proyectos existentes.

\item  Licencia de c\'{o}digo abierto en algunos casos. En ocasiones, las organizaciones necesitan organizar un repositorio interno en su propio servidor en lugar de utilizar los recursos web p\'{u}blicos existentes. El \'{u}nico de los tres repositorios que es de c\'{o}digo abierto es GitLab.
\end{enumerate}

\noindent 


\end{document}

